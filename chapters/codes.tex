\chapter{Кодын хэрэгжүүлэлт}


\begin{lstlisting}[language=JavaScript, frame=single, caption={Next.js Auth ашигласан байдал}]
import NextAuth, { AuthOptions } from "next-auth";
import GoogleProvider from "next-auth/providers/google";
import GithubProvider from "next-auth/providers/github";
import CredentialsProvider from "next-auth/providers/credentials";

export const authOptions: AuthOptions = {
  providers: [
    CredentialsProvider({
      id: "coldbrains-credentials",
      name: "Coldbrains",
      credentials: {
        username: { label: "Username", type: "text" },
        password: { label: "Password", type: "password" },
      },
      async authorize(credentials, req) {
        const reqBody = {
          username: credentials?.username,
          password: credentials?.password,
        };

        const res = await fetch(`${process.env.APP_URL}/api/login`, {
          method: "POST",
          body: JSON.stringify(reqBody),
        });
        if (res.status === 200) {
          const userCred = await res.json();
          return userCred;
        }
        return null;
      },
    }),
    GoogleProvider({
      clientId: process.env.GOOGLE_CLIENT_ID!,
      clientSecret: process.env.GOOGLE_CLIENT_SECRET!,
    }),
    GithubProvider({
      clientId: process.env.GITHUB_CLIENT_ID!,
      clientSecret: process.env.GITHUB_CLIENT_SECRET!,
    }),
  ],
  session: {
    strategy: "jwt",
  },
  pages: {
    signIn: "/login",
    signOut: "/problems",
  },
};

const handler = NextAuth(authOptions);

export { handler as GET, handler as POST };

\end{lstlisting}


\begin{lstlisting}[language=JavaScript, frame=single, caption={Codemirror буюу @uiw/react-codemirror-ийн компонентийг ашиглаж буй байдал}]
  "use client";
  import CodeMirror from "@uiw/react-codemirror";
  import {
    aura,
    abyss,
    androidstudio,
    dracula,
    material,
    copilot,
    githubDark,
    githubLight,
    sublime,
    xcodeDark,
    xcodeLight,
  } from "@uiw/codemirror-themes-all";
  import { python } from "@codemirror/lang-python";
  import { useState, useEffect } from "react";
  import styles from "./page.module.css";
  import { javascript } from "@codemirror/lang-javascript";
  import { oneDark } from "@uiw/react-codemirror";
  import { Language, Theme } from "./states/enum";
  import {
    Box,
    Select,
    MenuItem,
    FormControl,
    InputLabel,
    FormHelperText,
  } from "@mui/material";
  import { SelectChangeEvent } from "@mui/material/Select";
  
  export default function CodeEditor({
    boilerplate,
    problemId,
    problemTitle,
  }: {
    boilerplate: string | undefined;
    problemId: string;
    problemTitle: string;
  }) {
    const [pythonCode, setPythonCode] = useState(boilerplate || "");
    const [output, setOutput] = useState("");
  
    const [language, setLanguage] = useState("Python");
    const [editorTheme, setEditorTheme] = useState(Theme.aura);
    const [themeExt, setThemeExt] = useState(aura);
  
    const handleLanguageChange = (event: SelectChangeEvent) => {
      setLanguage(event.target.value);
    };
  
    const handleThemeChange = (event: SelectChangeEvent) => {
      setEditorTheme(Number(event.target.value));
      switch (Number(event.target.value)) {
        case Theme.abyss:
          setThemeExt(abyss);
          break;
        case Theme.androidstudio:
          setThemeExt(androidstudio);
          break;
        case Theme.aura:
          setThemeExt(aura);
          break;
        case Theme.dracula:
          setThemeExt(dracula);
          break;
        case Theme.githubDark:
          setThemeExt(githubDark);
          break;
        case Theme.githubLight:
          setThemeExt(githubLight);
          break;
        case Theme.material:
          setThemeExt(material);
          break;
        case Theme.oneDark:
          setThemeExt(oneDark);
          break;
        case Theme.copilot:
          setThemeExt(copilot);
          break;
        case Theme.sublime:
          setThemeExt(sublime);
          break;
        case Theme.xcodeDark:
          setThemeExt(xcodeDark);
          break;
        case Theme.xcodeLight:
          setThemeExt(xcodeLight);
          break;
        default:
          setThemeExt(aura);
          break;
      }
    };
  
    const handlePythonOnChange = (newPythonCode: string) =>
      setPythonCode(newPythonCode);
    const handlePythonCodeRun = async () => {
      const body = {
        id: problemId,
        code: pythonCode,
        language: language,
        title: problemTitle,
      };
      const response = await fetch("/api/run", {
        method: "POST",
        body: JSON.stringify(body),
      });
  
      console.log(response.status);
      if (response.status !== 202) {
        const data = await response.json();
        console.log(data);
        setOutput(data.traceback ?? "");
      } else {
        setOutput("Accepted");
      }
    };
  
    return (
      <Box>
        <Box
          sx={{
            bgcolor: "var(--primaryBlack)",
            borderTopLeftRadius: "15px",
            borderTopRightRadius: "15px",
            boxShadow: "0 0 10px 5px rgb(0, 0, 0, 0.3)",
          }}
        >
          <Box display={"flex"} justifyContent={"space-between"}>
            <Select
              variant="standard"
              value={language}
              onChange={handleLanguageChange}
              sx={{
                my: 1,
                mx: 2,
                px: 1,
                width: "120px",
                color: "#BBBBBB",
                bgcolor: "#21202e",
                borderRadius: "5px",
                ".MuiSvgIcon-root ": {
                  fill: "white",
                },
              }}
            >
              <MenuItem value={"Python"}>Python</MenuItem>
              <MenuItem value={"JS"}>Javascript</MenuItem>
              <MenuItem value={"CPP"}>C/C++</MenuItem>
            </Select>
            <Select
              variant="standard"
              value={editorTheme.toString()}
              onChange={handleThemeChange}
              sx={{
                my: 1,
                mx: 2,
                px: 1,
                width: "120px",
                color: "#BBBBBB",
                bgcolor: "#21202e",
                borderRadius: "5px",
                ".MuiSvgIcon-root ": {
                  fill: "white",
                },
              }}
            >
              <MenuItem value={Theme.abyss}>abyss</MenuItem>
              <MenuItem value={Theme.androidstudio}>androidstudio</MenuItem>
              <MenuItem value={Theme.aura}>aura</MenuItem>
              <MenuItem value={Theme.copilot}>copilot</MenuItem>
              <MenuItem value={Theme.dracula}>dracula</MenuItem>
              <MenuItem value={Theme.githubDark}>githubDark</MenuItem>
              <MenuItem value={Theme.githubLight}>githubLight</MenuItem>
              <MenuItem value={Theme.material}>material</MenuItem>
              <MenuItem value={Theme.oneDark}>oneDark</MenuItem>
              <MenuItem value={Theme.sublime}>sublime</MenuItem>
              <MenuItem value={Theme.xcodeDark}>xcodeDark</MenuItem>
              <MenuItem value={Theme.xcodeLight}>xcodeLight</MenuItem>
            </Select>
          </Box>
          <CodeMirror
            value={pythonCode}
            onChange={handlePythonOnChange}
            height="450px"
            extensions={[python()]}
            theme={themeExt}
          />
        </Box>
  
        <div className={styles.btn_submit} onClick={handlePythonCodeRun}>
          Run
        </div>
        <div className={styles.target_output}>
          <pre>{output}</pre>
        </div>
      </Box>
    );
  }
  
\end{lstlisting}

\begin{lstlisting}[language=JavaScript, frame=single, caption={NEXTjs бодлогууд авах серверийн endpoint}]
  import { NextRequest, NextResponse } from "next/server";
  import firestore from "@/configs/firebase";
  import { getDocs, collection } from "firebase/firestore";
  
  
  export async function GET(request: NextRequest) {
    const querySnapshot = await getDocs(collection(firestore, "problems"));
  
    const problemsArray = querySnapshot.docs.map((doc) => {
      return {
        id: doc.id,
        ...doc.data()
      }
    });
  
    return NextResponse.json(problemsArray);
  }
\end{lstlisting}

\begin{lstlisting}[language=JavaScript, frame=single, caption={Бодлогуудыг cache-ээс fetch хийх component}]
  "use client";
  import MainLayout from "@/components/layouts/mainLayout/mainLayout";
  import styles from "./page.module.css";
  import CodeMirrorTest from "@/components/editor/codeEditor";
  import ProblemList from "@/components/problemList/problemList";
  import { useSession } from "next-auth/react";
  import { useRouter } from "next/navigation";
  import { useEffect, useState } from "react";
  
  export default function Problems() {
    const [data, setData] = useState([]);
    const { status } = useSession();
    const router = useRouter();
  
    const columns = [
      { field: "title", headerName: "Бодлого", flex: 1, minWidth: 300 },
      { field: "status", headerName: "Төлөв", width: 150 },
      { field: "difficulty", headerName: "Түвшин", width: 150 },
    ];
  
    useEffect(() => {
      fetch("/api/getProblems")
        .then((res) => res.json())
        .then((resJson) => {
          setData(resJson);
        });
    }, []);
  
    const handleRowClick = (params: any) => {
      if (status === "authenticated") {
        router.push(`/problem/${params.row.id}`);
      } else {
        router.push("/login");
      }
    };
  
    return (
      <MainLayout>
        <ProblemList
          data={data}
          columns={columns}
          onRowClick={handleRowClick}
          rowsPerPage={12}
        />
      </MainLayout>
    );
  }
  
    
\end{lstlisting}

\begin{lstlisting}[language=JavaScript, frame=single, caption={CollectionGroup query бичиж хэрэглэгчийн бодолтуудыг авч буй route}]
import { getSession } from "@/providers/client-provider";
import { NextRequest, NextResponse } from "next/server";
import firestore from "@/configs/firebase";
import { getDocs, collectionGroup } from "firebase/firestore";

export async function GET(request: NextRequest) {
  const sesdata = await getSession();
  const email = sesdata?.user?.email;

  const querySnapshot = await getDocs(collectionGroup(firestore, "solutions"));
  const userSolutionDocs = querySnapshot.docs.filter((doc) => doc.id === email);
  const userSolutions = userSolutionDocs.map((userDoc) => userDoc.data());

  return NextResponse.json(userSolutions);
}

\end{lstlisting}

\begin{lstlisting}[language=JavaScript, frame=single, caption={Бодолт явуулах NEXTjs endpoint}]
import { NextRequest, NextResponse } from "next/server";
import { getSession } from "@/providers/client-provider";

export async function POST(request: NextRequest) {
  const { id, code, language, title } = await request.json();

  const sesdata = await getSession();
  if (!sesdata) {
    return new NextResponse("Forbidden", { status: 403 });
  }
  const userCred = sesdata?.user;

  switch (language) {
    case "Python":
      fetch(process.env.PROD_PYTHON_SERVER!);
      break;
    default:
      break;
  }

  // const res = await fetch(`${process.env.DEV_SERVER}/${language}`, {
  const res = await fetch(`${process.env.PROD_SERVER}/${language}`, {
    method: "POST",
    cache: "no-cache",
    headers: new Headers({ "content-type": "application/json" }),
    body: JSON.stringify({ id, code, userCred, title }),
  });

  return new NextResponse(res.body, { status: res.status });
}
    
\end{lstlisting}

\begin{lstlisting}[language=JavaScript, frame=single, caption={Хэрэглэгчийн нэрнээс Avatar үүсгэж буй байдал}]
import { Avatar } from "@mui/material";

const stringToColor = (string: string) => {
    let hash = 0;
    let i;
  
    /* eslint-disable no-bitwise */
    for (i = 0; i < string.length; i += 1) {
      hash = string.charCodeAt(i) + ((hash << 5) - hash);
    }
  
    let color = "#";
  
    for (i = 0; i < 3; i += 1) {
      const value = (hash >> (i * 8)) & 0xff;
      color += `00${value.toString(16)}`.slice(-2);
    }
    /* eslint-enable no-bitwise */
  
    return color;
};

type Size = "sm" | "md" | "lg";

const stringAvatarProps = (name: string, size?: Size) => {
  const fontSize = size === "sm" ? 18 : size === "lg" ? 60 : 20;

  return {
    sx: {
      backgroundColor: stringToColor(name),
      width: size === "sm" ? 30 : size === "lg" ? 100 : 40,
      height: size === "sm" ? 30 : size === "lg" ? 100 : 40,
      fontSize: fontSize,
    },
    children: `${name.charAt(0).toUpperCase()}`,
  };
};

const StringAvatar = ({ name, size }: { name: string; size?: Size }) => {
    return <Avatar {...stringAvatarProps(name, size)} />;
}

export default StringAvatar;
\end{lstlisting}


\begin{lstlisting}[language=JavaScript, frame=single, caption={Python хэлний middleware}]
import { Router } from "express";
import getFirebaseDB from "../db/firebase.js";
import { doc, getDoc, serverTimestamp, setDoc } from "firebase/firestore";
import axios from "axios";
import "dotenv/config";

const router = Router();

router.post("/", async (req, res) => {
  const code = req.body.code;
  const problemId = req.body.id;
  const problemTitle = req.body.title;
  const { name: username, email, image } = req.body.userCred;

  if (!code || !problemId) {
    return res.sendStatus(409);
  }

  const docRef = doc(getFirebaseDB(), "testCases", problemId);
  const docSnapshot = await getDoc(docRef);

  if (!docSnapshot.exists()) {
    return res.sendStatus(404);
  }

  const testCases = docSnapshot.get("testCase");
  const reqBody = {
    code,
    testCases,
  };

  try {
    // const axiosRes = await axios.post("http://localhost:8081", reqBody);
    const axiosRes = await axios.post(process.env.PROD_PYTHON_SERVER, reqBody);

    if (axiosRes.status === 202) {
      const docFields = {
        code,
        createdAt: serverTimestamp(),
        language: "Python",
        time: axiosRes.data,
        username,
        problemId,
        problemTitle,
        ...(image ? {image} : {})
      };

      const solutionDocRef = doc(
        getFirebaseDB(),
        "problemSolutions",
        problemId,
        "solutions",
        email
      );
      await setDoc(solutionDocRef, docFields);
    }
    return res.status(axiosRes.status).json(axiosRes.data);
  } catch (err) {
    return res.sendStatus(503);
  }
});

export default router;

\end{lstlisting}

\begin{lstlisting}[language=JavaScript, frame=single, caption={Stateless байдлаар firebase admin SDK ашиглалт}]
import { initializeApp } from "firebase/app";
import { getFirestore } from "firebase/firestore";
import "dotenv/config";

let db;

const firebaseConfigs = {
  apiKey: process.env.FIREBASE_API_KEY,
  authDomain: process.env.FIREBASE_AUTH_DOMAIN,
  databaseURL: process.env.FIREBASE_DATABASE_URL,
  projectId: process.env.FIREBASE_PROJECT_ID,
  storageBucket: process.env.FIREBASE_STORAGE_BUCKET,
  messagingSenderId: process.env.FIREBASE_MESSAGING_SENDER_ID,
  appId: process.env.FIREBASE_APP_ID,
  measurementId: process.env.FIREBASE_MEASUREMENT_ID,
};

export default function getFirebaseDB() {
  if (!db) {
    console.log("Initializing Firebase...");
    const app = initializeApp(firebaseConfigs);
    db = getFirestore(app);
    console.log("Firebase initialized.");
  }
  return db;
}
\end{lstlisting}

\begin{lstlisting}[language=JavaScript, frame=single, caption={ExpressJS серверийн оролт}]
import express from "express";
import cors from "cors";
import { rateLimit } from "express-rate-limit";

import getFirebaseDB from "./db/firebase.js";
import pythonRouter from "./routes/python.js";

const app = express();
const port = 8080;
const limiter = rateLimit({
  windowMs: 1000 * 15,
  limit: 5,
  message: "Too many requests. Please wait for 5 seconds and try again.",
})

app.use(cors());
app.use(express.json());
app.use(limiter)

app.use("/python", pythonRouter);
app.use("/js", null)
app.use("/c", null)

app.get("/", (req, res) => {
  res.sendStatus(200);
})

app.listen(port, () => {
  console.log("Python server listening.");
});

\end{lstlisting}