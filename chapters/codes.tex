\chapter{Кодын хэрэгжүүлэлт}


\begin{lstlisting}[language=JavaScript, frame=single, caption={Next.js Auth ашигласан байдал}]
import NextAuth, { AuthOptions } from "next-auth";
import GoogleProvider from "next-auth/providers/google";
import GithubProvider from "next-auth/providers/github";
import CredentialsProvider from "next-auth/providers/credentials";

export const authOptions: AuthOptions = {
  providers: [
    CredentialsProvider({
      id: "coldbrains-credentials",
      name: "Coldbrains",
      credentials: {
        username: { label: "Username", type: "text" },
        password: { label: "Password", type: "password" },
      },
      async authorize(credentials, req) {
        const reqBody = {
          username: credentials?.username,
          password: credentials?.password,
        };

        const res = await fetch(`${process.env.APP_URL}/api/login`, {
          method: "POST",
          body: JSON.stringify(reqBody),
        });
        if (res.status === 200) {
          const userCred = await res.json();
          return userCred;
        }
        return null;
      },
    }),
    GoogleProvider({
      clientId: process.env.GOOGLE_CLIENT_ID!,
      clientSecret: process.env.GOOGLE_CLIENT_SECRET!,
    }),
    GithubProvider({
      clientId: process.env.GITHUB_CLIENT_ID!,
      clientSecret: process.env.GITHUB_CLIENT_SECRET!,
    }),
  ],
  session: {
    strategy: "jwt",
  },
  pages: {
    signIn: "/login",
    signOut: "/problems",
  },
};

const handler = NextAuth(authOptions);

export { handler as GET, handler as POST };

\end{lstlisting}


\begin{lstlisting}[language=JavaScript, frame=single, caption={Хэрэглэгчийн компонент дээр дата render хийж буй байдал}]
"use client";
import CodeEditor from "@/components/editor/codeEditor";
import MainLayout from "@/components/layouts/mainLayout/mainLayout";
import { Box, Grid, Typography } from "@mui/material";
import { use } from "react";

async function getData() {
  try {
    const res = await fetch(
      "https://api.sampleapis.com/codingresources/codingResources"
    );
    if (res.ok) {
      return res.json();
    }
  } catch (e) {
    return [];
  }

  throw new Error("Failed to fetch data");
}

export default async function Problem({ params }: any) {
  const data = await getData();
  const data = [
    {},
    {
      title: "Эрэмбэлсэн хоёр дарааллын медиан",
      description:
        "Lorem ipsum dolor sit amet consectetur, adipisicing elit. Maxime et, aperiam accusamus quisquam odio laborum ipsa vero atque saepe tempora earum iste sapiente? Blanditiis quas quasi, unde aliquam nam alias!",
      boilerplate: "def addTwoNumbers(num1, num2):",
    },
  ];
  const problemId = params.id;

  return (
    <MainLayout>
      <Box width={"100%"} mt={2}>
        <Typography variant="h4">{data[problemId].title}</Typography>
      </Box>
      <Grid container spacing={2} my={1}>
        <Grid item md={5} xs={12} bgcolor={"var(--primaryBlack)"} borderRadius={"20px"}>
          <Typography variant="body2">{data[problemId].description}</Typography>
        </Grid>
        <Grid item md={7} xs={12}>
          <CodeEditor boilerplate={data[problemId].boilerplate} />
        </Grid>
      </Grid>
    </MainLayout>
  );
}

\end{lstlisting}

\begin{lstlisting}[language=JavaScript, frame=single, caption={MUI ашиглаж pagination хийж буй жишээ}]
"use client";
import React, { useState } from "react";
import { Pagination, Box } from "@mui/material";

export default function ProblemPagination({ page, onPageChange, pageCount }: any) {
  const handlePageChange = (
    event: React.ChangeEvent<unknown>,
    newPage: number
  ) => {
    onPageChange(newPage);
  };

  return (
    <Pagination
      count={pageCount}
      page={page}
      onChange={handlePageChange}
      color="primary"
      sx={{
        '& .MuiPaginationItem-root': {
          color: "white",
          backgroundColor: "var(--primaryBlack)",
        },
        '& .MuiPaginationItem-page.Mui-selected': {
          backgroundColor: "brown",
          color: "white"
        },
      }}
    />
  );
}

\end{lstlisting}

\begin{lstlisting}[language=JavaScript, frame=single, caption={ExpressJS auth сервер}]
import express from "express";
import "dotenv/config";
import bodyParser from "body-parser";
import cors from "cors";

import loginRoute from "./routes/login.js";
import registerRoute from "./routes/register.js";

const PORT = process.env.APP_PORT;
const app = express();

app.use(cors());
app.use(bodyParser.json());

app.use("/auth/login", loginRoute);
app.use("/auth/register", registerRoute)

app.listen(PORT, () => {
  console.log(`Coldbrians auth server is listening on: ${PORT}`);
})

\end{lstlisting}

\begin{lstlisting}[language=JavaScript, frame=single, caption={python-shell санг ашиглаж буй байдал}]
const express = require("express");
const cors = require("cors");

const pythonShell = require("python-shell");
const vm = require("vm");

const app = express();
const port = 8080;

app.use(cors());
app.use(express.text());

app.post("/python", (req, res) => {
    const code = req.body;

    pythonShell.PythonShell.runString(code, null)
    .then((messages) => {
        console.log(messages);
        res.json({ passOrFail: messages.join() });
    })
    .catch((err) => {
        res.json({
        passOrFail:
            "Your code has issues. Please use print() method to output.",
        });
    });
}); 

app.post("/javascript", (req, res) => {
    const code = req.body;

    const sandbox = { result: null };

    const script = new vm.Script(code);
    const context = new vm.createContext(sandbox);
    script.runInContext(context);
    res.json({ passOrFail: sandbox.result });
});

app.listen(port, () => {
    console.log("Python server listening.");
});
    
\end{lstlisting}
