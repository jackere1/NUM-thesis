%----------------------------------------------------------------------------------------
%   Доорх хэсгийг өөрчлөх шаардлагагүй
%----------------------------------------------------------------------------------------
%!TEX TS-program = xelatex
%!TEX encoding = UTF-8 Unicode
\documentclass[12pt,A4]{report}

\usepackage{fontspec,xltxtra,xunicode}
\setmainfont[Ligatures=TeX]{Times New Roman}
\setsansfont{Arial}

% \usepackage[utf8x]{inputenc}
% \usepackage[mongolian]{babel}
%\usepackage{natbib}
\usepackage{geometry}
%\usepackage{fancyheadings} fancyheadings is obsolete: replaced by fancyhdr. JL
\usepackage{fancyhdr}
\usepackage{float}
\usepackage{afterpage}
\usepackage{graphicx}
\usepackage{amsmath,amssymb,amsbsy}
\usepackage{dcolumn,array}
\usepackage{tocloft}
\usepackage{dics}
\usepackage{nomencl}
\usepackage{upgreek}
\newcommand{\argmin}{\arg\!\min}
\usepackage{mathtools}
\usepackage[hidelinks]{hyperref}

\usepackage{algorithm}
\usepackage{algpseudocode}

\usepackage{listings}
\DeclarePairedDelimiter\abs{\lvert}{\rvert}%
\makeatletter
\usepackage{caption}
\captionsetup[table]{belowskip=0.5pt}
\usepackage{subfiles}
\usepackage[table,xcdraw]{xcolor}

\usepackage{listings}
\renewcommand{\lstlistingname}{Код}
\renewcommand{\lstlistlistingname}{\lstlistingname ын жагсаалт}
\renewcommand{\bibname}{Ном зүй}

\usepackage{color}
\definecolor{codegreen}{rgb}{0,0.6,0}
\definecolor{codegray}{rgb}{0.5,0.5,0.5}
\definecolor{codepurple}{rgb}{0.58,0,0.82}
\definecolor{backcolour}{rgb}{0.99,0.99,0.99}
 
\lstdefinestyle{mystyle}{
    basicstyle=\ttfamily\small,
    backgroundcolor=\color{backcolour},   
    commentstyle=\color{codegreen},
    keywordstyle=\color{magenta},
    numberstyle=\tiny\color{codegray},
    stringstyle=\color{codepurple},
    %basicstyle=\footnotesize,
    breakatwhitespace=false,         
    breaklines=true,                 
    captionpos=b,                    
    keepspaces=false,                 
    numbers=left,                    
    numbersep=10pt,                  
    showspaces=false,                
    showstringspaces=true,
    showtabs=false,                  
    tabsize=2
}

\lstset{style=mystyle, label=DescriptiveLabel} 
\lstdefinelanguage{JavaScript}{
	keywords={typeof, new, true, false, catch, function, return, null, catch, switch, var, if, in, while, do, else, case, break, import, export, throw, new},
	keywordstyle=\color{blue}\bfseries,
	ndkeywords={class, export, boolean, throw, implements, import, this},
	ndkeywordstyle=\color{darkgray}\bfseries,
	identifierstyle=\color{black},
	sensitive=false,
	comment=[l]{//},
	morecomment=[s]{/*}{*/},
	commentstyle=\color{purple}\ttfamily,
	stringstyle=\color{red}\ttfamily,
	morestring=[b]',
	morestring=[b]"
}
	 
\lstset{
    language=JavaScript,
	extendedchars=true,
	basicstyle=\footnotesize\ttfamily,
	showstringspaces=false,
	showspaces=false,
	numbers=left,
	numberstyle=\footnotesize,
	numbersep=9pt,
	tabsize=2,
	breaklines=true,
	showtabs=false,
	captionpos=b
}

\let\oldabs\abs
\def\abs{\@ifstar{\oldabs}{\oldabs*}}
\makenomenclature
\begin{document}


%----------------------------------------------------------------------------------------
%   Өөрийн мэдээллээ оруулах хэсэг
%----------------------------------------------------------------------------------------

% Дипломийн ажлын сэдэв
\title{Алгоритм Болон Програмчлалын Аргууд Сурах Интерактив Систем}
% Дипломын ажлын англи нэр
\titleEng{Interactive Algorithm and Programming Learning System}
% Өөрийн овог нэрийг бүтнээр нь бичнэ
\author{Нямдоржийн Энхболд}
% Өөрийн овгийн эхний үсэг нэрээ бичнэ
\authorShort{Н.Энхболд}
% Удирдагчийн зэрэг цол овгийн эхний үсэг нэр
\supervisor{Дэд проф. Б.Сувдаа}
% Хамтарсан удирдагчийн зэрэг цол овгийн эхний үсэг нэр
% \cosupervisor{Др. Г.Амарсанаа}

% СиСи дугаар 
\sisiId{20B1NUM0102}
% Их сургуулийн нэр
\university{МОНГОЛ УЛСЫН ИХ СУРГУУЛЬ}
% Бүрэлдэхүүн сургуулийн нэр
\faculty{МЭДЭЭЛЛИЙН ТЕХНОЛОГИ, ЭЛЕКТРОНИКИЙН СУРГУУЛЬ}
% Тэнхимийн нэр
\department{МЭДЭЭЛЭЛ, КОМПЬЮТЕРИЙН УХААНЫ ТЭНХИМ}
% Зэргийн нэр
\degreeName{Бакалаврын судалгааны ажил}
% Суралцаж буй хөтөлбөрийн нэр
\programeName{Програм хангамж (D061302)}
% Хэвлэгдсэн газар
\cityName{Улаанбаатар}
% Хэвлэгдсэн огноо
\gradyear{2023 он}


%----------------------------------------------------------------------------------------
%   Доорх хэсгийг өөрчлөх шаардлагагүй
%----------------------------------------------------------------------------------------
%----------------------Нүүр хуудастай хамаатай зүйлс----------------------------
\pagenumbering{roman}
\makefrontpage
\maketitle

\doublespace

% Decleration
\begin{huge}
\textbf{Зохиогчийн баталгаа}
\end{huge} \\ \ \\ 
\doublespace
Миний бие \@author \ "\@title" \ сэдэвтэй судалгааны ажлыг гүйцэтгэсэн болохыг зарлаж дараах зүйлсийг баталж байна:
\begin{itemize}
\item Ажил нь бүхэлдээ эсвэл ихэнхдээ Монгол Улсын Их Сургуулийн зэрэг горилохоор дэвшүүлсэн болно.
\item Энэ ажлын аль нэг хэсгийг эсвэл бүхлээр нь ямар нэг их, дээд сургуулийн зэрэг горилохоор оруулж байгаагүй.
\item Бусдын хийсэн ажлаас хуулбарлаагүй, ашигласан бол ишлэл, зүүлт хийсэн.
\item Ажлыг би өөрөө (хамтарч) хийсэн ба миний хийсэн ажил, үзүүлсэн дэмжлэгийг дипломын ажилд тодорхой тусгасан. 
\item Ажилд тусалсан бүх эх сурвалжид талархаж байна. 
\end{itemize} 
\ 

Гарын үсэг: \underline{\hspace{5cm}} 

Огноо: 	\ \ \underline{\hspace{3cm}}

% Гарчгийг автоматаар оруулна
\setcounter{tocdepth}{1}
\tableofcontents

% Зургийн жагсаалтыг автоматаар оруулна
\listoffigures

% Хүснэгтийн жагсаалтыг автоматаар оруулна
\listoftables

% Кодын жагсаалтыг автоматаар оруулна
\lstlistoflistings

% This puts the word "Page" right justified above everything else.
\newpage
%% \addtocontents{lof}{Зураг~\hfill Хуудас \par}
\newpage
%% \addtocontents{lot}{Хүснэгт~\hfill Хуудас \par}

\renewcommand{\cftlabel}{Зураг}


\doublespace
\pagenumbering{arabic}


% Удиртгалыг оруулж ирэх ба abstract.tex файлд удиртгалаа бичнэ
% \chapter{Удиртгал}
\begin{abstract}
  Програмчлалын тэр дундаа алгоритмын түгээмэл бодлого, асуудлуудыг шийдэх нь компьютерийн ухааны чиглэлээр сурагчид болон програм хангамж хөгжүүлэгчдийн үндсэн суурь чадвар болж өгдөг. Цаашлаад тухайн алгоритмуудыг төрөл бүрийн програмчлалын хэл болон архитектур, зохиомж зэрэг бодит жишээн дээр хэрэгжүүлэх шаардлага түгээмэл гардаг.

  Энэхүү судалгааны ажлаар хийсэн бүтээл нь "Coldbrains" гэх веб технологиудад суурилсан, програмчлалд суралцах сонирхолтой хэрэглэгчид рүү чиглэсэн програмчлалын бодлого бодох систем юм. Энэхүү систем нь хэрэглэгчийн програмчлал болон алгоритмын мэдлэгийг бататгаж, шийдлээ бусадтайгаа хуваалцах боломжийг бий болгоно. Цаашлаад бодлого бүрт харгалзах бодолтын санг бий болгох билээ.
\end{abstract}

%----------------------------------------------------------------------------------------
%   Дипломын үндсэн хэсэг эндээс эхэлнэ
%----------------------------------------------------------------------------------------
\addcontentsline{toc}{part}{БҮЛГҮҮД}
% Шинэ бүлэг
\subfile{chapters/introduction.tex}

\subfile{chapters/systemResearch.tex}

\subfile{chapters/architecture.tex}

\subfile{chapters/design.tex}

\subfile{chapters/implementation.tex}

%----------------------------------------------------------------------------------------
%   Дүгнэлт эндээс эхэлнэ
%----------------------------------------------------------------------------------------
\conclusion{Дүгнэлт}
...

%----------------------------------------------------------------------------------------
%   Дипломын номзүй, хавсралтын хэсэг эндээс эхэлнэ
%----------------------------------------------------------------------------------------

\singlespace
\addcontentsline{toc}{part}{НОМ ЗҮЙ}
\begin{thebibliography}{}
    \bibitem{motherLanguage}
    Benefits of Learning in Mother Tongue language \url{https://www.krsf.in/blog/benefits-of-learning-in-mother-tongue-language/}
    \bibitem{wireframe}
    Coldbrains, miro workspace \url{https://miro.com/app/board/uXjVNf5zADk=/?share_link_id=375550425121}
    \bibitem{VSC theme}
    Most popular VS Code themes (2021) \url{https://visualstudiomagazine.com/articles/2021/07/07/vs-code-themes.aspx}
    \bibitem{acceptedmn}
    Accepted academy 2023 оны 12 сар \url{https://www.accepted.mn/it}
    \bibitem{spojmn}
    Spoj - "Алгоритмын цагаан толгой" бодлогын сан \url{https://www.spoj.com/RGB7}
    \bibitem{awsLambda}
    AWS Lambda - Programming Model \url{https://docs.aws.amazon.com/lambda/latest/dg/foundation-progmodel.html}
    \bibitem{codemirrorinfo}
    Codemirror - Extensible code editor \url{https://codemirror.net}
    \bibitem{bcrypt}
    Understanding bcrypt. \url{https://auth0.com/blog/hashing-in-action-understanding-bcrypt}
    \bibitem{datamodeling}
    Firestore - Data Modeling \url{https://firebase.google.com/docs/firestore/data-model}
    \bibitem{firestoreindex}
    Firestore - Index types \url{https://firebase.google.com/docs/firestore/query-data/index-overview}
\end{thebibliography}


%----------------------------------------------------------------------------------------
%   Хавсралтууд эндээс эхэлнэ
%----------------------------------------------------------------------------------------
\appendix
\addcontentsline{toc}{part}{ХАВСРАЛТ}

% Хавсралтын нэр. Хавсралт гэдэг үг агуулахгүй
\subfile{chapters/codes.tex}

\end{document}